	\documentclass{report}
\usepackage[table,xcdraw]{xcolor}
\usepackage[utf8]{inputenc}
\usepackage{geometry}
\usepackage{amsmath}
\usepackage{amsthm}
\usepackage{amsfonts}
\usepackage{amssymb}
\usepackage{graphicx}
\usepackage{tocloft}
\usepackage{pgfplots}
\usepackage{multirow}	

\usepackage[usenames,dvipsnames]{pstricks}
\usepackage{epsfig}
\usepackage{pst-grad} % For gradients
\usepackage{pst-plot} % For axes
\usepackage[space]{grffile} % For spaces in paths
\usepackage{etoolbox} % For spaces in paths

%\usepackage{perpage} %the perpage package
%\MakePerPage{footnote} %the perpage package command

% For adding TOC in pdf bookmarks
\usepackage{hyperref}
\hypersetup{pdftex,colorlinks=true,allcolors=black}
\usepackage{hypcap}
%
\usepackage{float}
\usepackage{listings}
\usepackage{color}

\definecolor{codegreen}{rgb}{0,0.6,0}
\definecolor{codegray}{rgb}{0.5,0.5,0.5}
\definecolor{codepurple}{rgb}{0.58,0,0.82}
\definecolor{backcolour}{rgb}{0.95,0.95,0.92}

\lstdefinestyle{mystyle}{
	backgroundcolor=\color{backcolour},   
	commentstyle=\color{codegreen},
	keywordstyle=\color{magenta},
	numberstyle=\tiny\color{codegray},
	stringstyle=\color{codepurple},
	basicstyle=\footnotesize,
	breakatwhitespace=false,         
	breaklines=true,                 
	captionpos=b,                    
	keepspaces=true,                 
	numbers=left,                    
	numbersep=5pt,                  
	showspaces=false,                
	showstringspaces=false,
	showtabs=false,                  
	tabsize=2
}

\lstset{style=mystyle}
\usepackage{xepersian}
\usepackage{bidi}

\settextfont{Yas}
\SepMark{-}

\renewcommand{\cftsecleader}{\cftdotfill{\cftdotsep}}

\theoremstyle{definition}
\newtheorem{definition}{تعریف}

\title{مستندات پروژه درس هوش مصنوعی و سیستم‌های خبره}
\author{امیر حقیقتی ملکی}
\date{بهار ۹۶}
	
\begin{document}
	%%%%%%%%%%%%%%%%%%%%%%%%%%%%%%%
	%%	 TITLE PAGE - BEGIN	     %%
	%%%%%%%%%%%%%%%%%%%%%%%%%%%%%%%
	\newgeometry{margin=1in}
	\pagenumbering{gobble}
		\begin{titlepage}
		\centering
		\includegraphics[width=0.25\textwidth]{../../../Template/Resources/logo.png}\par\vspace{1cm}
		{\scshape\LARGE دانشگاه صنعتی امیرکبیر \par}
		{\scshape\LARGE دانشکده مهندسی کامپیوتر و فناوری اطلاعات \par}
		\vspace{1cm}
		{\scshape\Large 
			مستندات پروژه درس
			«هوش مصنوعی و سیستم‌های خبره»
			\par}
		\vspace{1.5cm}
		{\huge\bfseries 
			پروژه اول
			\par}
		\vspace{2cm}
		{\Large امیر حقیقتی ملکی\par}
		{\Large ۹۳۳۱۰۰۹\par}
		\vfill
		استاد درس:\par
		دکتر نیک‌آبادی
		\vfill
		
		% Bottom of the page
		{\large \rl{
				بهار ۹۶
			}\par}
	\end{titlepage}
	\newpage
	\pagenumbering{gobble}
	\tableofcontents
	\newpage
	\pagenumbering{arabic}
	\chapter{ساختار کلی و چارچوب}
	\section{مقدمه}
	پروژه اینجانب با نام
	GS-ProblemSolver
	است که مخفف عبارت حل‌کننده مسئله با جستجوی گراف (هر نوع گرافی از جمله درخت) است.\\
	این پروژه به زبان Java نوشته شده؛ همچنین، در پیاده‌سازی این چارچوب سعی شده‌است تا جای ممکن، به صورت کلی طراحی و پیاده‌سازی انجام گیرد تا کاربر هنگام کار با این چارچوب حل مسئله، دیگر نگران جزئیات پیاده‌سازی نباشد.\\
	برای استفاده از این چارچوب، می‌بایست ابتدا کلاس
	Problem
	را تعریف کرد و سپس توسط جستجوگرهای مختلفی که وجود دارند، عمل جستجو را انجام داد. توضیحات هر کلاس در ادامه ذکر شده است.\\
	ساختار فایل‌های این پروژه عبارت است از:
	\begin{latin}
			\begin{itemize}
				\item src.main
				\subitem Main.java
				\item src.resources
				\subitem Action.java
				\subitem Graph.java
				\subitem GSException.java
				\subitem Node.java
				\subitem Problem.java
				\item src.searchers
				\subitem AstarSearcher.java
				\subitem BDSearcher.java
				\subitem BFSearcher.java
				\subitem DFSearcher.java
				\subitem Searcher.java
				\subitem UCSearcher.java
			\end{itemize}
	\end{latin}
	کلاس‌های جستجوکننده که در بسته searchers موجود می‌باشند حاوی الگوریتم‌ها و روش‌هایی برای حل مسائل مختلف می‌باشند که نهایتا با استفاده از این کلاس‌ها می‌توانیم به جواب برسیم.
	کلاس Problem نیز برای هر مسئله می‌بایست تعریف شود. سپس می‌توان با استفاده از فراخوانی تابع search جستجوگرها، به دنبال پاسخ گشت.
\end{document}
