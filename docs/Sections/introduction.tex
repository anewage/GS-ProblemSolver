\section{مقدمه}
	عامل‌های هوشمند واکنشی ساده، در محیط‌هایی که تعداد حالات آن زیاد است و در مسائلی که به مسائل جهان واقعی نزدیک است نمی‌توانند بازدهی مناسبی داشته باشند و در نتیجه دیگر سادگی آن‌ها کارآمد نخواهد بود. برای حل مسائل واقعی، می‌توانیم از عامل‌های مبتنی بر هدفی که اعمال و اقدامات خود را پیش از انجام مورد سنجش قرار می‌دهند استفاده کنیم. این عامل‌ها، حالت‌های محیط خود را به دید اتمیک نگاه می‌کنند؛ به این معنی که در هر حالت، حالت فعلی دیگر قابل تقسیم به جزء های کوچک‌تر نیست. در این‌جا به منظور سادگی فرض می‌شود که راه‌حل یک مسئله، همواره دنباله‌ای ثابت از عملیات مختلف است؛ عمل‌های عامل در طول جستجو برای جواب، در اثر ورود ادراکات مختلف عوض نخواهند شد. با این مفروضات، روش‌ها و الگوریتم‌هایی که برای جستجوی راه‌حل در مسائلی از این قبیل در نظر گرفته می‌شوند می‌توانند از نوع آگاهانه یا ناآگاهانه باشند که در ادامه به توضیح آن‌ها پرداخته شده است.
	\subsection{عامل و محیط}
	عامل‌‌ها می‌توانند با دریافت ادراکات از محیط، با انجام عمل‌هایی روی محیط تاثیر گذاشته و حالت آن‌را عوض کنند و یا اینکه حالت کنونی خود در مقایسه با محیط را تغییر دهند. طبق تعریف عامل‌های هوشمند، می‌بایست فعالیت‌های این عامل‌ها منجر به بیشینه شدن معیار کارآیی\footnote{Performance measure} آن‌ها شود. گاهی اوقات، عامل می‌تواند با داشتن هدفی برای خود و تلاش برای رسیدن به آن هدف، معیار کارآیی خود را آسان‌تر به میزان بیشینه برساند. اهداف می‌توانند با محدود مقاصد عامل از انجام عمل‌های مختلف، در سامان‌دهی رفتار عامل کمک به سزایی داشته باشند. در نتیجه فرموله‌کردن هدف و مقصد مورد نظر - براساس موقعیت و حالت کنونی عامل و معیار کارآیی آن - اولین قدم برای حل مسئله می‌باشد.\\
	\paragraph{هدف}
	برای ما، مجموعه‌ای از چند حالت محیط می‌باشد که عامل می‌تواند با انجام عمل‌های مختلف خود را در آن حالت(ها) قرار دهد و به مقصود خود دست یابد؛ برای این منظور عامل می‌بایست مشخص کند که چه عملیاتی باید انجام دهد.\\
	پیش از تصمیم درباره عمل خاصی، عامل باید بداند که باید چه نوع عمل‌هایی را برای انجام و چه حالت‌هایی از محیط را برای رسیدن باید در نظر بگیرد. فرموله‌کردن مسئله به معنی تعریف و توصیف دقیق حالت‌ها و عملیات ممکن با توجه به حالت‌های هدف در یک محیط است. سطح تجرید عملیات‌ها و همچنین میزان جزئیات توصیف حالت‌های محیط می‌تواند بسته به مسئله متفاوت باشد و باید حتما به این نکته توجه شود که ذکر جزئیات بیش از حد و یا کم‌تر از میزان قابل قبول می‌تواند به عدم موفقیت عامل در دستیابی به پاسخ و رسیدن به هدف شود.
	\paragraph{محیط}
	برای تصمیم‌گیری در مورد انجام عملیات مختلف، عامل می‌بایست ابتدا بداند که در حالت کنونی خود مجاز به انجام چه عملیاتی است و نتیجه هر عمل منجر به رسیدن به کدام حالت از محیط می‌شود؛ سپس با بررسی هر عمل و با در نظر داشتن هدف، می‌تواند بهترین تصمیم را بگیرد. بنابراین فرض می‌شود که محیط
	\textbf{کاملا قابل مشاهده}
	است. همچنین در صورت نامحدود بودن عملیات قابل انجام در هر حالت، امکان تصمیم‌گیری برای عامل در عمل وجود نخواهد داشت. در نتیجه محیط می‌بایست
	\textbf{گسسته}
	باشد. به این معنی که در هر حالت، تعداد محدودی عمل قابل انجام است. همچنین در فرآیند تصمیم‌گیری، عامل مورد نظر ما می‌بایست قادر به پیش‌بینی نتیجه حاصل از عمل خود باشد. در غیر این‌صورت قادر به تصمیم‌گیری نخواهد بود. به عبارت دیگر فرض می‌شود محیط
	\textbf{شناخته شده}
	است و عامل می‌داند که با انجام هر عمل از حالت فعلی به کدام حالت می‌تواند برود. همچنین عدم قطعیت اعمال به کلی نادیده گرفته می‌شود و هر عامل در صورت انجام عملی، حتما و الزاما خروجی مربوط به آن عمل را خواهد دید. بنابراین محیط
	\textbf{قطعی}
	است.
	\paragraph{جستجو}
	فرآیند یافتن دنباله‌ای از عملیات که منجر به رسیدن به هدف شود جستجوی راه‌حل نام دارد. یک الگوریتم جستجوی راه‌حل، مسئله‌ای (فضای حالت)‌ را به عنوان ورودی گرفته و راه‌حل آن مسئله را در قالب دنباله‌ای از عملیات و اقدامات توصیف کرده و برمی‌گرداند.الگوریتم‌های مختلفی برای جستجوی راه‌حل در محیط‌های گوناگون وجود دارد که در ادامه به توضیح آن‌ها می‌پردازیم.\\
	هنگامی که راه‌حلی پیدا شد، می‌توان دنباله عملیات ذکر شده در راه‌حل را انجام داد؛ به این مرحله، مرحله اجرا گفته می‌شود. بنابراین برای رسیدن به هدف، می‌بایست ابتدا فرموله‌بندی مناسبی از مسائل و محیط ارائه دهیم، سپس راه‌حل را جستجو کنیم و در مرحله آخر، راه‌حل به دست آمده را برای اجرا به عامل بدهیم.\\
	\subsection{ساختار مسائل و راه‌حل‌ها}
	همانطور که در قسمت قبل ذکر شد، به دلیل گرافی بودن فضای حالت مسئله، برای فرموله کردن هر مسئله، گره‌های گراف را در قالب داده‌ساختاری به نام گره ذخیره می‌کنیم که این داده‌ساختار دارای مشخصه‌های زیر است:
	\begin{enumerate}
		\item
		آدرس گره پدر:
		برای یافتن مسیری که منتهی به این گره شده می‌بایست آدرس گره پدر را در هر گره ذخیره کنیم. این مشخصه در یافتن دنباله حالت‌های جستجو شده در راه‌حل مناسب می‌باشد.
		\item
		حالت عامل:
		حالتی که عامل با رسیدن به گره فعلی در آن قرار می‌گیرد در این داده‌ساختار ذخیره می‌شود.
		\item
		عمل انجام شده:
		عملی که با انجام آن به گره فعلی رسیده‌ایم و گره فعلی در نتیجه انجام آن عمل در حالت گره پدر می‌باشد. این مشخصه برای به دست آوردن دنباله عملیات انجام شده در طی یک راه‌حل مناسب می‌باشد.
		\item
		هزینه مسیر تا این گره:
		هزینه‌ای که تا این رسیدن به این گره پرداخت شده است. این مولفه برای به دست آوردن هزینه مسیر راه‌حل مناسب است.
	\end{enumerate}
	 همچنین به دلیل گرافی بودن فضای حالت مسئله، می‌بایست برای هر مسئله موارد زیر را تعریف کنیم:
	\begin{enumerate}
		\item
		ساختار کلی حالت‌ها:
		اینکه حالت‌ها در قالب ماتریس، عدد، رشته و یا غیره بیان می‌شوند، می‌بایست توسط مسئله بیان شود.
		\item
		حالت اولیه:
		حالت اولیه در واقع نشان دهنده پیکربندی\footnote{configuration} اولیه و حالت شروع عامل می‌باشد. این حالت در جستجوی راه‌حل به عنوان گره آغازین در نظر گرفته می‌شود و در راه‌حل پیدا شده نیز اولین گام خواهد بود.
		\item
		حالت هدف:
		درواقع اینکه عامل در مورد مسئله مورد نظر به حالت درستی رسیده یا نه توسط این حالت مشخص می‌شود. مقصد نهایی عامل می‌بایست اینجا باشد. در مسائلی که نتوان حالت هدف را به طور دقیق بیان کرد می‌بایست شرایط حالت هدف ذکر شود.
		\item
		نتیجه هر عمل در حالت‌های مختلف:
		از آن‌جا که فرض کرده‌ایم که محیط ما کاملا قابل مشاهده است، مسئله باید گویای نتیجه هر عمل (حالت حاصل شده در ازای انجام عمل) در هر حالتی از فضای حالت مسئله باشد.
		\item
		عملیات ممکن در هر حالت:
		پیرو مورد قبلی، از آن‌جا که فرض کرده‌ایم که محیط ما کاملا قابل مشاهده است، مسئله باید گویای عملیات قابل انجام توسط عامل در هر حالتی از فضای حالت باشد.
		\item
		هزینه انجام هر عمل در هر حالت:
		عامل باید بداند که هزینه انجام هر عمل در هر حالتی که باشد چقدر خواهد بود.
		\item
		هزینه مسیری که تا کنون پرداخت شده:
		عامل باید بداند که تا کنون و تا رسیدن به حالت فعلی چه هزینه‌ای پرداخت کرده است.
		\item
		تابع شهودی هزینه:
		این تابع که در واقع تخمینی است از هزینه باقی‌مانده تا رسیدن به حالت هدف، در صورتی باید پیاده‌سازی شود که بخواهیم از روش‌های جستجوهای آگاهانه در یافتن پاسخ استفاده کنیم.
	\end{enumerate}
در هر پیمایش و جستجوی راه‌حل، گرهیی ساخته می‌شود که پدر آن گره فعلی بوده و تمامی خصوصیت‌های نام برده شده نیز برای آن ست می‌شوند. اگر در حین جستجوی راه‌حل به گرهی برسیم که رمقصد مورد نظر عامل باشد، برای به دست آوردن دنباله اعمال انجام شده از حالت اولیه، کافی است از گره انتهایی (مقصد) که در جستجوی راه‌حل ساخته شده، شروع کرده و تا زمانی که به گره مبدا نرسیده‌ایم دنباله اعمال و حالت‌ها را بسازیم.
	
	
	
	
	

	
	
	
	
	
	
	
	