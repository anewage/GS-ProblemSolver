	\begin{abstract}
	حل مسئله با جستجوی درخت و در حالت کلی گراف فضای حالت مسئله یکی از روش‌های کلاسیک برای حل مسائلی است که در آن محیط کاملا قابل مشاهده، گسسته، شناخته شده و قطعی می‌باشد. در این روش، عامل‌های مبتنی بر هدف توصیف می‌شوند و با دادن فرموله‌بندی مناسبی از مسئله مورد نظر، می‌توان به راه‌حل (در صورت وجود) دست یافت. طبق تعریف پروژه، فریم‌ورکی\footnote{framework} برای این منظور پیاده‌سازی شد که در آن از الگوریتم‌های جستجوی گرافی اول سطح، اول عمق (عمق نامحدود، عمق محدود، افزایش تدریجی عمق)، هزینه یکنواخت، دو جهته که از جزو الگوریتم‌های جستجوی ناآگاهانه به حساب آمده و A* که یک الگوریتم جستجوی آگاهانه است، استفاده شد. این فریم‌ورک به زبان Java بوده و مستندات مربوط به استفاده از آن و نحوه فرموله‌بندی مسئله در این گزارش ذکر شده است. سه مسئله نمونه مسیریابی، پازل لغزشی و مبلغان مذهبی و آدم‌خوار‌ها در تعریف پروژه ذکر شده‌اند که با انجام فرموله‌بندی و انجام آزمایشات، پاسخ این مسائل و نتایج عملکرد هرکدام از الگوریتم‌های پیاده‌سازی شده مشخص شدند. الگوریتم جستجوی گرافی اول بهترین حریصانه نیز می‌تواند در آینده به منظور افزودن قابلیت‌های بیشتر به این فریم‌ورک اضافه شود.
	\end{abstract}
	\keywords{
		هوش مصنوعی، مسائل کلاسیک، حل مسئله با جستجو، جستجوی آگاهانه، جستجوی ناآگاهانه،
		جستجوی گراف
	}